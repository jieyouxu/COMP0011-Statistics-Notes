\section{Discrete Random Variables}

\subsection{Definitions}

\begin{definition}[Random Variable]
    A \textit{random variable} $v$ maps each point in the sample space $\Omega$
    to a real number.

    \begin{equation}
        v \colon \Omega \to \Real
    \end{equation}

    Random variables have two classifications:

    \begin{enumerate}
        \item \textbf{Discrete} random variables; and
        \item \textbf{Continuous} random variables.
    \end{enumerate}
\end{definition}

\begin{definition}[Discrete Random Variable]
    A \textit{discrete} random variable can take on values from a countable set
    of potential values.
\end{definition}

\begin{example}[Coin Flip]
    Let $X$ take on the values:

    \begin{equation*}
        X =
        \begin{cases}
            0 & \text{if $X$ is head} \\
            1 & \text{if $X$ is tail}
        \end{cases}
    \end{equation*}

    Then $X$ is a discrete random variable as it takes on either of the two
    values in $\set{0, 1}$.
\end{example}

\begin{definition}[Continuous Random Variable]
    A \textit{continuous} random variable can take on values from an
    uncountable set of potential values.
\end{definition}

\begin{example}[Rainfall]
    Let $R$ denote the amount of rainfall in one day in mm.

    Then $R$ can take on values from $\PosReal$, i.e. a continuous random
    variable.
\end{example}

\subsection{Sample Measures}

\begin{definition}[Relative Frequency Distribution]
    We can derive \textit{relative frequency distribution} for discrete data
    collected from the \textit{sample}.

    Let $n = \sum \limits_{i = 1}^{n} f_i$ be the sample size, then

    \begin{center}
        \begin{tabular}{c|c}
            Observed values & Relative frequency \\
            \hline
            $x_1$ & $f_1 / n$ \\
            $\vdots$ & $\vdots$ \\
            $x_N$ & $f_N / n$ \\
        \end{tabular}
    \end{center}
\end{definition}

\begin{definition}[Sample Mean]
    The \textit{sample mean} $\bar{x}$ is defined as

    \begin{equation}
        \sum \limits_{i=1}^{N} x_i \frac{f_i}{n}
    \end{equation}
\end{definition}

\begin{definition}[Sample Variance]
    The \textit{sample variance} $s^2$ is defined as

    \begin{align}
        s^2
        &= \sum \limits_{i=1}^{N} \left(
            x_i - \bar{x} \right)^2 \frac{f_i}{n - 1} \\
        &= \frac{\left( \sum
            \limits_{i=1}^{N} x_i^2 f_i - n \bar{x}^2 \right)}{n - 1}
    \end{align}
\end{definition}

\begin{definition}[Sample Standard Deviation]
    The \textit{sample standard deviation} is given by

    \begin{equation}
        s = \sqrt{s^2}
    \end{equation}
\end{definition}

\subsection{Population Measures}

\begin{definition}[Probability Distribution]
    We can derive \textit{probability distribution} with respect to a random
    variable $X$ for the \textit{population}

    \begin{center}
        \begin{tabular}{c|c}
            Possible values & Probability \\
            \hline
            $x_1$ & $p_1 = \Prob{X = x_1}$ \\
            $\vdots$ & $\vdots$ \\
            $x_N$ & $p_N = \Prob{X = x_N}$ \\
        \end{tabular}
    \end{center}

    With the important properties:

    \begin{enumerate}
        \item Each possibility $p_i$ is \textit{non-negative}.
            \begin{equation}
                \Forall i \in [1, n] \colon p_i \ge 0
            \end{equation}
        \item The possibility distribution sum to $1$.
            \begin{equation}
                \sum \limits_{i = 1}^{N} p_i = 1
            \end{equation}
    \end{enumerate}
\end{definition}

\begin{definition}[Population Mean]
    The \textit{population mean} $\mu$ is defined as

    \begin{equation}
        \mu = \sum \limits_{i=1}^{N} x_i p_i = \ExpVal{X}
    \end{equation}

    Where $\ExpVal{X}$ denotes the \textit{expected value} for $X$.
\end{definition}

\begin{definition}[Population Variance]
    The \textit{population variance} $\sigma^2$ is defined as

    \begin{align}
        \sigma^2
        &= \sum \limits_{i=1}^{N} (x_i - \mu)^2 p_i \\
        &= \sum \limits_{i=1}^{N} x_i^2 p_i - \mu^2 \\
        &= \Var{X}
    \end{align}

    Where $\Var{X}$ denotes the \textit{variance} of $X$.
\end{definition}

\begin{definition}[Population Standard Deviation]
    The \textit{population standard deviation} $\sigma$ is given by

    \begin{equation}
        \sigma = \sqrt{\sigma^2}
    \end{equation}
\end{definition}

\subsection{Expectation}

\begin{definition}[Expectation (Expected Value)]
    Let $X$ be a random variable and let $Y = f(X)$ be some function
    $f \colon X \to Y$, then the expected value $\ExpVal{Y}$ of $Y$ is given as

    \begin{align}
        \ExpVal{Y}
        &= \ExpValS{f(x)} \\
        &= \sum \limits_{i=1}^{N} f(x_i) \Prob{X = x_i} \\
        &= \sum \limits_{i=1}^{N} f(x_i) p_i
    \end{align}

    Note that

    \begin{equation}
        \ExpVal{Y} \equiv \sum_j y_j \Prob{Y = y_j}
    \end{equation}
\end{definition}

\begin{definition}[Variance]
    Since

    \begin{align*}
        \Var{X}
        &= \sum \limits_{i=1}^{N} (x_i - \mu)^2 p_i \\
        &= \ExpValS{(X - \mu)^2} \\
        &= \sum \limits_{i=1}^{N} x_i^2 p_i - \mu^2 \\
        &= \ExpValS{X^2} - \mu^2
    \end{align*}

    then

    \begin{equation}
        \Var{X} = \ExpValS{X^2} - \mu^2
    \end{equation}
\end{definition}

\subsection{Independence, Expectation and Variance}

\begin{remark}[Event Independence]
    Two events $E_1$ and $E_2$ are \textit{independent} if and only if

    \begin{equation*}
        \Prob{E_1 \cap E_2} = \Prob{E_1} \Prob{E_2}
    \end{equation*}
\end{remark}

\begin{definition}[Independent Discrete Random Variables]
    Two \textit{discrete random variables} $X$ and $Y$ are \textit{independent}
    if and only if

    \begin{equation}
        \Forall x_i \in X \colon \Forall y_j \in Y \colon
            \Prob{X = x_i \cap Y = y_j} = \Prob{X = x_i} \Prob{Y = y_j}
    \end{equation}

    That is, each pair of events $\set{ X = x_i }$ and $\set{ Y = y_j }$ are
    \textit{independent}.
\end{definition}

\begin{definition}[Expectation of Discrete Random Variable Plus Constant]
    If $X$ is a discrete random variable, and $b \in \Real$ is some constant,
    then

    \begin{equation}
        \ExpVal{X + b} = \ExpVal{X} + b
    \end{equation}

    Because this is a shift for each occurrence $x_i$.
\end{definition}

\begin{definition}[Variance of Discrete Random Variable Plus Constant]
    If $X$ is a discrete random variable, and $b \in \Real$ is some constant,
    then

    \begin{equation}
        \Var{X + b} = \Var{X}
    \end{equation}

    With constant $b$ ignored because shifting each expected value by some
    constant do not change how their spread is.
\end{definition}

\begin{definition}[Expectation of Discrete Random Variable Times Constant]
    If $X$ is a discrete random variable, and $a \in \Real$ is some constant,
    then

    \begin{equation}
        \ExpVal{aX} = a\ExpVal{X}
    \end{equation}

    Because each occurrence $x_i$ is scaled by $a$, causing the mean to also
    scale by $a$.
\end{definition}

\begin{definition}[Variance of Independent Discrete Random Variable Times Constant]
    If $X$ is an independent discrete random variable, and $a \in \Real$ is
    some constant, then

    \begin{equation}
        \Var{aX} = a^2 \Var{X}
    \end{equation}

    Because smaller samples get scaled a smaller absolute amount while larger
    samples get scaled a larger absolute amount (even if they are scaled at
    the same proportion $a$), causing the spread to increase.
\end{definition}

\begin{definition}[Expectation of a Linear Transformation]
    Let independent discrete random variable $Y = aX + b$, where $X$ is an
    independent discrete random variable, then the expectation of $Y$ is given
    as

    \begin{equation}
        \ExpVal{Y} = \ExpVal{aX + b} = a\ExpVal{X} + b
    \end{equation}
\end{definition}

\begin{definition}[Variance of a Linear Transformation]
    Let independent discrete random variable $Y = aX + b$, where $X$ is an
    independent discrete random variable, then the variance of $Y$ is given as

    \begin{equation}
        \Var{Y} = \Var{aX + b} = a^2 \Var{X}
    \end{equation}
\end{definition}

\begin{definition}[Expectation for Two Independent Discrete Random Variables Added]
    If $X, Y$ are two \textit{independent} discrete random variables, then the
    expectation $\ExpVal{X + Y}$ is given as

    \begin{equation}
        \ExpVal{X + Y} = \ExpVal{X} + \ExpVal{Y}
    \end{equation}
\end{definition}

\begin{definition}[Expectation for Two Independent Discrete Random Variables Subtracted]
    If $X, Y$ are two \textit{independent} discrete random variables, then the
    expectation $\ExpVal{X - Y}$ is given as

    \begin{equation}
        \ExpVal{X - Y} = \ExpVal{X} - \ExpVal{Y}
    \end{equation}
\end{definition}

\begin{definition}[Variance for Two Independent Discrete Random Variables Added]
    If $X, Y$ are two \textit{independent} discrete random variables, then the
    expectation $\Var{X + Y}$ is given as

    \begin{equation}
        \Var{X + Y} = \Var{X} + \Var{Y}
    \end{equation}

    Note that this does \textit{not} hold if one of $X$ and $Y$ is dependent
    on the other!
\end{definition}

\begin{definition}[Variance for Two Independent Discrete Random Variables Added]
    If $X, Y$ are two \textit{independent} discrete random variables, then the
    expectation $\Var{X - Y}$ is given as

    \begin{equation}
        \Var{X - Y} = \Var{X} + \Var{Y}
    \end{equation}

    Note that their variance is still \textit{added}!
\end{definition}

\subsection{Discrete Distributions}

\begin{definition}[Distributions for a Random Variable]
    For some random variable $X$, we use the tilde to describe its random
    distribution (named $\mathrm{DistributionName}$)

    \begin{equation}
        X \sim \mathrm{DistributionName}(\;
            \langle \mathit{parameter} \rangle, \dots \;)
    \end{equation}
\end{definition}

\subsubsection{Bernoulli Distribution}

\begin{definition}[Bernoulli Distribution]
    The \textit{Bernoulli distribution} is suitable for an experiment with
    discrete \textit{binary} outcomes, either \textit{success} or
    \textit{failure}, respectively labelled as $n = 0$ (\enquote{failure}) or
    $n = 1$ (\enquote{success}).

    \begin{itemize}
        \item The \textit{success} case $n = 1$ occurs with probability $p$.
        \item The \textit{failure} case $n = 0$ occurs with probability $q
            \equiv 1 - p$.
        \item $0 < p < 1$.
    \end{itemize}

    The \textit{probability mass function} (PMF) for $X \sim \Bernoulli{p}$
    with success probability $p$ is

    \begin{equation}
        \Prob{n} =
        \begin{cases}
            1 - p & \text{if $n = 0$} \\
            p & \text{if $n = 1$} \\
        \end{cases}
    \end{equation}

    The \textit{expected value} for $X \sim \Bernoulli{p}$ is

    \begin{equation}
        \ExpVal{X} = p
    \end{equation}

    The \textit{variance} for $X \sim \Bernoulli{p}$ is

    \begin{equation}
        \Var{X} = p - p^2 = p(1 - p) = pq
    \end{equation}
\end{definition}

\subsubsection{Binomial Distribution}

\begin{remark}
    A \textit{binomial distribution} is the sum of \textit{identically}
    and \textit{independently} distributed \textit{Bernoulli} random variables,
    with each Bernoulli random variable having the outcome of either $0$ or
    $1$.
\end{remark}

\begin{definition}[Binomial Distribution]
    The \textit{binomial distribution} is suitable for a repeated sequence of
    \textit{independent} and \textit{identical} random experiments, with each
    experiment having binary outcomes (so each random experiment has Bernoulli
    distribution). Each binary Bernoulli experiment has success probability
    $p$ and failure probability $q \equiv 1 - p$. The random variable
    $X \sim \Binomial{n}{p}$ is the number of \textit{successes} in $n$
    independent repeated experiments, each with success probability $p$.

    The \textit{probability mass function} (PMF) for getting $k$ successes in
    $n$ independent Bernoulli trials is given as

    \begin{equation}
        \Prob{X = k} = \binom{n}{k} p^k (1 - p)^{n - k}
    \end{equation}

    For each possible success count $k = 0, 1, \dots, n$, with $\binom{n}{k}$
    being the binomial coefficient

    \begin{equation}
        \binom{n}{k} = \frac{n!}{k!(n-k)!}
    \end{equation}

    For $i = 1, \dots, n$, let

    \begin{equation}
        X_i = \begin{cases}
            0 & \text{$i$th repetition is failure} \\
            1 & \text{$i$th repetition is success} \\
        \end{cases}
    \end{equation}

    Each of $X_1, \dots, X_n$ is an independent and identical
    $X_i \sim \Bernoulli{p}$ random variable, hence
    $\ExpVal{X_i} = p$ and $\Var{X_i} = pq = p(1 - p)$.

    Since $X = \sum \limits_{i=1}^n X_i$, then the \textit{expected value} for
    $X \sim \Binomial{n}{p}$ is then

    \begin{equation}
        \ExpVal{X} = \sum \limits_{i=1}^n \ExpVal{X_i} = np
    \end{equation}

    And the \textit{variance} for $X \sim \Binomial{n}{p}$ is then

    \begin{equation}
        \Var{X} = \sum \limits_{i=1}^n \ExpVal{X_i} = npq = np(1-p)
    \end{equation}
\end{definition}

\begin{remark}
    The Binomial distribution is \textit{not} suitable for cases where
    experiments are \textit{not independent}. For example, taking black versus
    red balls out of a container \textit{without replacement} changes the
    probability of getting a black ball after each experiment.
\end{remark}

\subsubsection{Geometric Distribution}

\begin{definition}[Geometric Distribution]
    The \textit{geometric distribution} is suitable for \textit{independent}
    repetitions of a \textit{binary} experiment, with each experiment having
    success probability $p$ and failure probability $q \equiv 1 - p$.

    The random variable $X \sim \Geometric{p}$ describes the number of trials
    \textit{up to and including} the \textit{first success}. That is, how many
    tries are required to get the first success (including the first successful
    try).

    The probability that the $k$th trial is the first success ($k - 1$ failures
    followed by the $k$th success) is

    \begin{equation}
        \Prob{X = k} = pq^{k - 1} = p(1-p)^{k-1}, \quad k = 1, 2, \dots
    \end{equation}

    The \textit{expected value} for $X \sim \Geometric{p}$ is

    \begin{equation}
        \ExpVal{X} = \frac{1}{p}
    \end{equation}

    The \textit{variance} for $X \sim \Geometric{p}$ is

    \begin{equation}
        \Var{X} = \frac{q}{p^2} = \frac{1 - p}{p^2}
    \end{equation}
\end{definition}

\subsubsection{Poisson Distribution}

\begin{remark}
    The \textit{Poisson distribution} is suitable for approximating the
    binomial distribution in the case of large $n$ and small $p$ (rule of
    thumb: $n \ge 20$ and $p \le 0.05$).
\end{remark}

\begin{definition}[Poisson Distribution]
    The \textit{Poisson distribution} is useful for:

    \begin{enumerate}
        \item Modelling the number of random events that occurs during a given
            \textit{time interval} or \textit{space interval}.
        \item Approximating a binomial distribution with large $n$ ($n \ge 20$)
            and small $p$ ($p \le 0.05$).
    \end{enumerate}

    The probability of observing $k$ events in the given interval, with the
    \textit{rate parameter} $\lambda$ is

    \begin{equation}
        \Prob{X = k} = \frac{e^{-\lambda} \lambda^k}{k!}
    \end{equation}

    Then $X \sim \Poisson{\lambda}$ for \textit{rate parameter} $\lambda$.

    The \textit{expected value} of $X \sim \Poisson{\lambda}$ is exactly its
    rate parameter $\lambda$

    \begin{equation}
        \ExpVal{X} = \lambda
    \end{equation}

    The \textit{variance} of $X \sim \Poisson{\lambda}$ is also exactly its
    rate parameter $\lambda$

    \begin{equation}
        \Var{X} = \lambda
    \end{equation}
\end{definition}

\begin{definition}[Assumptions of the Poisson Process]
    The \textit{Poisson process} assumes that:

    \begin{enumerate}
        \item Events occur \textit{randomly}.
        \item Events occur \textit{one at a time}.
        \item Events occur at the \textit{average rate} of $\lambda$ per unit time or per unit space.
        \item Number of observed events in non-overlapping time or space intervals are \textit{independent}.
    \end{enumerate}

    If $X$ is the number of events during a \textit{time interval} or \textit{space interval} of length $t$, then $X \sim \Poisson{\lambda}$ has the \textit{mean} $\mu = \lambda t$.
\end{definition}

\begin{definition}[Approximating Binomial Distribution with Poisson Distribution]
    For a binomial distribution $X \sim \Binomial{n}{p}$ with $n \ge 20$ and $p \le 0.05$ then we can \textit{approximate} it with a Poisson distribution with $\lambda = np$.

    \begin{equation}
        X \sim \Binomial{n}{p}, \; n \ge 20 \land p \le 0.05 \implies X \ApproxDistributed \Poisson{\lambda}, \; \lambda = n p
    \end{equation}

    Where $\ApproxDistributed$ denotes that the distribution is an approximation.
\end{definition}
