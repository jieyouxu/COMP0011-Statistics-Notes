\section{Data Representation}

\subsection{Terminology}

\begin{definition}[Random Experiment]
    Action with uncertain outcome.
\end{definition}

\begin{definition}[Population]
    The set of all possible replications of the experiment.
    
    \begin{equation*}
        \text{Population} \coloneqq \set{ \text{all possible replications of the experiment} }
    \end{equation*}
\end{definition}

\begin{definition}[Sample]
    A \textit{subset} of the population.
    
    \begin{equation*}
        \text{Sample} \subseteq \text{Population}
    \end{equation*}
\end{definition}

\begin{definition}[Statistical Inference]
    From the \textit{sample} we may be able to \textit{infer} conclusions about the \textit{population}.
    
    \begin{equation*}
        \text{Sample} \xRightarrow{\text{statistical inference}} \text{Population}
    \end{equation*}
\end{definition}

\begin{definition}[Variable]
    A \textit{quantity} or \textit{characteristic} of interest for any member of the population.
\end{definition}

\begin{example}\
    \begin{itemize}
        \item Age
        \item Time until failure
        \item Degree subject
    \end{itemize}
\end{example}

\begin{definition}[Data]
    The set of values of one or more variables recorded for a sample taken from a population.
\end{definition}

\subsection{Types of Data}

\begin{enumerate}
    \item \textbf{Discrete}: \textit{countable} possible set of values.
    \begin{enumerate}
        \item \textbf{Binary}: only 2 possible values.
        \item \textbf{Categorical}: only categories are possible values.
    \end{enumerate}
    
    \item \textbf{Continuous}: \textit{uncountable} set of possible values.
\end{enumerate}

\subsection{Graphs}

\begin{definition}[Bar Chart]
    Suitable for \textit{discrete} data.
    
    \begin{figure}[H]
        \centering
        \includegraphics
            [width=0.5\textwidth]
            {content/img/1-bar-chart.png}
        \caption{Bar chart.}
        \label{fig:bar-chart-example}
    \end{figure}
    
    \begin{itemize}
        \item \textit{Height} of each bar $\propto$ \textit{frequency} of the value
        \item Can also use \textit{relative frequencies} (proportions) instead
    \end{itemize}
\end{definition}

\begin{definition}[Histogram]
    Suitable for \textit{continuous} data.
    
    The data range can be \textit{partitioned} into \textit{class intervals}.
    
    \begin{figure}[H]
        \centering
        \includegraphics
            [width=0.5\textwidth]
            {content/img/1-histogram.png}
        \caption{Histogram.}
        \label{fig:histogram-example}
    \end{figure}

    \begin{itemize}
        \item \textit{Area} of each box $\propto$ \textit{frequency} of the class interval.
        \item Can also use the \textit{relative frequencies} (proportions) instead.
    \end{itemize}
\end{definition}

\subsection{Summary Statistics}

We can calculate \textit{summative measures} from data.

\begin{definition}[Summative Measures]\
    \begin{itemize}
        \item Measures of \textbf{location}:
        \begin{enumerate}
            \item Mean
            \item Median
            \item Mode
        \end{enumerate}
        \item Measures of \textbf{dispersion}:
        \begin{enumerate}
            \item Sample variance $\sigma^2$ (and sample standard deviation $\sigma$)
            \item Range
            \item Interquartile range
            \item Coefficient of variation ($r^2$)
        \end{enumerate}
    \end{itemize}
\end{definition}

\begin{definition}[Mean]
    One type of \textit{average}.
    
    Given $n$ \textit{observations}, each denoted as $x_i$, then the sample mean $\Bar{x}$ is defined by
    
    \begin{equation}\label{eq:mean}
        \Bar{x} = \frac{1}{n} \sum \limits_{i=1}^{n} x_i
    \end{equation}
    
    Given $n$ \textit{observed values}, with each observed value $x_i$ observed $f_i$ times (frequency $f_i$), then the sample mean $\Bar{x}$ is defined by
    
    \begin{equation}\label{eq:mean_freq}
        \Bar{x} = \frac{
            \sum \limits_{i=1}^{n} f_i x_i
        }{
            \sum \limits_{i=1}^{n} f_i
        }
    \end{equation}
\end{definition}
